\documentclass[12pt]{article}
\usepackage[UTF8]{ctex}     % 支持中文
\usepackage[utf8]{inputenc} % 编码
\usepackage{amsmath}        % 数学公式
\usepackage{amssymb}        % 数学符号
\usepackage{geometry}       % 页面调整
\geometry{a4paper, margin=1in}

\title{Generative Model Midterm Solutions}
\author{whoami}
\date{\today}

\begin{document}

\maketitle

% --- 第1题 (a) ---
\section*{Problem 1 (a): }

\textbf{Training Phase(训练阶段):}
\begin{itemize}
    \item RNN: The complexity is $O(n)$. 因为它是按顺序一个接一个算的 (Sequential)。
    \item Transformer: The complexity is $O(n^2)$. 因为 Self-Attention 要算两两之间的关系。
    \item \textbf{Conclusion:} Transformer is more training-efficient (RNN的第n个要等前面的第n-1个算完, Transformer可以并行计算).
\end{itemize}

\textbf{Inference Phase(推理阶段):}
\begin{itemize}
    \item RNN: Complexity is $O(n)$ total.
    \item Transformer: Complexity is $O(n^2)$ total.
    \item \textbf{Conclusion:} RNN is more inference-efficient .
\end{itemize}


% --- 第1题 (b) ---
\section*{Problem 1 (b): Positional Embeddings}

Self-Attention 机制本质上是将输入视为一个无序集合并行处理的,输入输出是排列等变的,只计算内容相似度而忽略距离,因此缺乏捕捉序列顺序(如语法结构)的内在能力,需要位置编码。


% --- 第1题 (c) ---
\section*{Problem 1 (c): Self-Attention Mechanism}

Given queries $\mathbf{q}$, keys $\mathbf{k}$, and values $\mathbf{v}$. The output is calculated as:

\[
    \text{Attention}(Q, K, V) = \text{softmax}\left( \frac{QK^T}{\sqrt{d}} \right) V
\]

\[
\mathbf{o}_i = \sum_{j=1}^n \alpha_{ij} \mathbf{v}_j, \quad \text{where} \quad 
\alpha_{ij} = \frac{\exp\left( \frac{\mathbf{q}_i^\top \mathbf{k}_j}{\sqrt{d}} \right)}{\sum_{k=1}^n \exp\left( \frac{\mathbf{q}_i^\top \mathbf{k}_k}{\sqrt{d}} \right)}
\]



% --- 第1题 (d) ---
\section*{Problem 1 (d): Layer Normalization}

题目给出的输入 $\mathbf{X}$ 是4个向量。LayerNorm 是对每一个向量单独做归一化 (mean=0, std=1)。

\vspace{0.5cm} % 这里空一点距离好看

\textbf{Step 1: Analyze Input}
For the first vector $x_1 = \begin{bmatrix} 2 \\ 1 \end{bmatrix}$:
\[
    \text{Mean } \mu = \frac{2+1}{2} = 1.5
\]
\[
    \text{Variance } \sigma^2 = \frac{(2-1.5)^2 + (1-1.5)^2}{2} = 0.25 \implies \sigma = 0.5
\]

\textbf{Step 2: Normalize}
\[
    \hat{x}_1 = \frac{x_1 - \mu}{\sigma} = \begin{bmatrix} (2-1.5)/0.5 \\ (1-1.5)/0.5 \end{bmatrix} = \begin{bmatrix} 1 \\ -1 \end{bmatrix}
\]

同理 (Similarly),对于其他向量计算后,最终结果都是 $\begin{bmatrix} 1 \\ -1 \end{bmatrix}$ 或 $\begin{bmatrix} -1 \\ 1 \end{bmatrix}$。

\textbf{Final Answer:}
\[
    \mathbf{X}_{out} = \left[ 
    \begin{bmatrix} 1 \\ -1 \end{bmatrix}, 
    \begin{bmatrix} -1 \\ 1 \end{bmatrix}, 
    \begin{bmatrix} 1 \\ -1 \end{bmatrix}, 
    \begin{bmatrix} 1 \\ -1 \end{bmatrix} 
    \right]
\]


\section*{Problem 2 (a): ELBO Proof (via KL Divergence)}

We derive the lower bound by analyzing the log-likelihood $\log p_\theta(\mathbf{x})$. Since $\int q_\phi(\mathbf{z}|\mathbf{x}) d\mathbf{z} = 1$, we can write:

\[
\begin{aligned}
    \log p_\theta(\mathbf{x}) 
    &= \mathbb{E}_{q_\phi(\mathbf{z}|\mathbf{x})} \left[ \log p_\theta(\mathbf{x}) \right] \\
    &= \mathbb{E}_{q_\phi(\mathbf{z}|\mathbf{x})} \left[ \log \frac{p_\theta(\mathbf{x}, \mathbf{z})}{p_\theta(\mathbf{z}|\mathbf{x})} \right] \quad (\text{Bayes Rule: } p(\mathbf{x}) = \frac{p(\mathbf{x},\mathbf{z})}{p(\mathbf{z}|\mathbf{x})}) \\
    &= \mathbb{E}_{q_\phi(\mathbf{z}|\mathbf{x})} \left[ \log \left( \frac{p_\theta(\mathbf{x}, \mathbf{z})}{q_\phi(\mathbf{z}|\mathbf{x})} \cdot \frac{q_\phi(\mathbf{z}|\mathbf{x})}{p_\theta(\mathbf{z}|\mathbf{x})} \right) \right] \quad (\text{Multiply by } \frac{q}{q}) \\
    &= \underbrace{\mathbb{E}_{q_\phi(\mathbf{z}|\mathbf{x})} \left[ \log \frac{p_\theta(\mathbf{x}, \mathbf{z})}{q_\phi(\mathbf{z}|\mathbf{x})} \right]}_{\text{ELBO}} + \underbrace{\mathbb{E}_{q_\phi(\mathbf{z}|\mathbf{x})} \left[ \log \frac{q_\phi(\mathbf{z}|\mathbf{x})}{p_\theta(\mathbf{z}|\mathbf{x})} \right]}_{D_{KL}(q_\phi(\mathbf{z}|\mathbf{x}) \| p_\theta(\mathbf{z}|\mathbf{x}))}
\end{aligned}
\]

Since the KL divergence is always non-negative ($D_{KL} \ge 0$):
\[
    \log p_\theta(\mathbf{x}) \ge \mathbb{E}_{q_\phi(\mathbf{z}|\mathbf{x})} \left[ \log \frac{p_\theta(\mathbf{x}, \mathbf{z})}{q_\phi(\mathbf{z}|\mathbf{x})} \right]
\]
Thus, the variational lower bound is proved.


\section*{Problem 2 (b): Reparameterization Trick}

\textbf{1. 什么是重参数化技巧 (What is the trick?)}

重参数化技巧是将随机变量 $\mathbf{z}$ 表示为确定性参数和独立噪声的函数。
我们不再直接从分布 $\mathcal{N}(\mu, \sigma^2)$ 中采样,而是引入一个标准高斯噪声 $\epsilon \sim \mathcal{N}(0, \mathbf{I})$,通过以下公式生成 $\mathbf{z}$:
\[
    \mathbf{z} = \mu + \sigma \odot \epsilon
\]
其中 $\mu$ 和 $\sigma$ 是 Encoder 网络的输出,$\odot$ 表示逐元素相乘。

\vspace{0.5cm}

\textbf{2. 为什么需要它 (Why do we need it?)}

正如题目背景所述,在 Encoder 部分,输入 $\mathbf{x}$ 被映射为隐空间分布的参数 $\mu$ 和 $\sigma$。

\begin{itemize}
    \item \textbf{直接采样的断路问题:} 如果我们直接从 $\mathcal{N}(\mu, \sigma^2)$ 中采样 $\mathbf{z}$,这是一个不可导的随机操作(Stochastic operation)。计算图(Computational Graph)在此处中断,导致 Loss 对 Encoder 参数(即对 $\mu$ 和 $\sigma$)的梯度无法通过反向传播(Backpropagation)传回去。
    
    \item \textbf{重参数化的解决之道:} 通过变换 $\mathbf{z} = \mu + \sigma \odot \epsilon$,所有的随机性被转移到了外部常量 $\epsilon$ 上。此时,对于网络参数 $\mu$ 和 $\sigma$ 而言,$\mathbf{z}$ 变成了一个平滑的可导函数(加法和乘法)。这打通了梯度流,使得神经网络可以进行端到端的训练。

\end{itemize}

% =========================================
% Problem 3: GAN
% =========================================
\section*{Problem 3: GAN (20 points)}

% --- 题目 3(a) ---
\section*{Problem 3 (a): Training and Inference Process}

GAN consists of two networks: a \textbf{Generator} $f_G$ (creates fake data from noise) and a \textbf{Discriminator} $f_D$ (distinguishes real data from fake).

\textbf{1. Training Process (Min-Max Game)} \\
The training is a zero-sum game where $f_D$ tries to maximize the discrimination accuracy, while $f_G$ tries to minimize it (fool the discriminator).
The objective function is:
\[
    \min_{G} \max_{D} V(D, G) = \mathbb{E}_{x \sim p_{\text{data}}} [\log D(x)] + \mathbb{E}_{z \sim p_z} [\log (1 - D(G(z)))]
\]

The process is usually iterative:
\begin{itemize}
    \item \textbf{Step 1 (Train Discriminator):} Sample a batch of real data $x$ and noise $z$. Fix the Generator's parameters. Update $f_D$ to \textbf{maximize} the probability of assigning 1 to real data and 0 to fake data $G(z)$.
    \item \textbf{Step 2 (Train Generator):} Sample a new batch of noise $z$. Fix the Discriminator's parameters. Update $f_G$ to \textbf{minimize} $\log(1 - D(G(z)))$ (or practically, maximize $\log D(G(z))$) to trick the discriminator.
\end{itemize}

\textbf{2. Inference Process} \\
During inference, the Discriminator $f_D$ is discarded. We only use the trained Generator $f_G$:
\[
    z \sim p_z \quad \xrightarrow{f_G} \quad \hat{x} = f_G(z)
\]
We sample random noise $z$ (e.g., from Gaussian) and pass it through $f_G$ to generate new samples.

% --- 题目 3(b) ---
\section*{Problem 3 (b): KL Divergence as f-divergence}

\textbf{Definition of f-divergence:}
Given two distributions $P$ and $Q$ with densities $p(x)$ and $q(x)$, the f-divergence is defined as:
\[
    D_f(P \| Q) = \int q(x) f\left( \frac{p(x)}{q(x)} \right) dx
\]
where $f(t)$ is a convex function with $f(1) = 0$.

\textbf{Proof:}
We need to find a specific function $f(t)$ such that $D_f$ becomes the KL divergence.
Let us choose the generator function:
\[
    f(t) = t \log t
\]

\textbf{1. Check convexity and constraint:}
\begin{itemize}
    \item Defined on the domain $t > 0$, non-negative.
    \item Derivative: $f'(t) = 1 + \log t$.
    \item Second derivative: $f''(t) = \frac{1}{t}$. Since $t$ represents a probability ratio $\frac{p(x)}{q(x)}$, $t > 0$, so $f''(t) > 0$. Thus, $f(t)$ is convex.
    \item Constraint: $f(1) = 1 \log 1 = 0$. Satisfied.
\end{itemize}

\textbf{2. Substitution:}
Substitute $f(t) = t \log t$ and $t = \frac{p(x)}{q(x)}$ into the f-divergence formula:
\[
\begin{aligned}
    D_f(P \| Q) &= \int q(x) \left[ \frac{p(x)}{q(x)} \log \left( \frac{p(x)}{q(x)} \right) \right] dx \\
    &= \int p(x) \log \frac{p(x)}{q(x)} dx \\
    &= D_{KL}(P \| Q)
\end{aligned}
\]
Thus, KL divergence is a special case of f-divergence where $f(t) = t \log t$.



\section*{Problem 4: Normalizing Flow (20 points)}

% --- 题目 4(a) ---
\section*{Problem 4 (a): Change of Variables Theorem}

Given the relationship $\mathbf{X} = g(\mathbf{Z})$ where $g$ is an invertible and smooth function, and $\mathbf{Z} \sim p_{\mathbf{Z}}(z)$.
According to the multivariate change of variables theorem, the probability density $p_{\mathbf{X}}(x)$ is:

\[
    p_{\mathbf{X}}(x) = p_{\mathbf{Z}}(z) \left| \det \left( \frac{\partial g^{-1}(x)}{\partial x} \right) \right|
\]
Alternatively, it can be written using the forward transformation Jacobian:
\[
    p_{\mathbf{X}}(x) = p_{\mathbf{Z}}(g^{-1}(x)) \cdot \left| \det \left( \frac{\partial g(z)}{\partial z} \right) \right|^{-1}
\]

\textbf{Intuition:}
Probability mass must be conserved ($p(x)dx = p(z)dz$). The term $\left| \det \frac{\partial z}{\partial x} \right|$ represents the change in volume (distortion) caused by the transformation.

% --- 题目 4(b) ---
\section*{Problem 4 (b): Designing Invertible MLP}

We define the layer as $g(\mathbf{x}) = \sigma(W\mathbf{x} + \mathbf{b})$. To ensure $g(\mathbf{x})$ is invertible and the Jacobian determinant is easy to compute, we impose the following constraints:

\begin{enumerate}
    \item \textbf{Weights $W$ (Constraint: Triangular):} \\
    We restrict $W$ to be an \textbf{Upper (or Lower) Triangular Matrix} with non-zero diagonal elements.
    \[
        W_{ij} = 0 \text{ if } i > j \quad (\text{Upper Triangular})
    \]
    \textit{Reason:} This makes the determinant simply the product of diagonal elements ($\det W = \prod w_{ii}$), and triangular matrices are invertible if diagonals are non-zero.

    \item \textbf{Activation $\sigma$ (Constraint: Strictly Monotonic):} \\
    We choose a strictly increasing function, such as \textbf{LeakyReLU} or \textbf{PReLU}.
    \[
        \sigma(y) = \begin{cases} y & \text{if } y > 0 \\ \alpha y & \text{if } y \le 0 \end{cases} \quad (\text{where } \alpha > 0)
    \]
    \textit{Reason:} Strictly monotonic functions are bijective (invertible). ReLU is not suitable because it is not invertible for negative values (maps to 0).
    
    \item \textbf{Bias $\mathbf{b}$:} No specific constraint (translation is always invertible).
\end{enumerate}

% --- 题目 4(c) ---
\section*{Problem 4 (c): Training and Inference Algorithm}

Since Normalizing Flows are generative models trained by \textbf{Maximum Likelihood Estimation (MLE)}.

\textbf{1. Training Objective (Exact Log-Likelihood)} \\
Our goal is to maximize the log-likelihood of the observed data $\mathbf{x}$.
Let $\mathbf{z} = g^{-1}(\mathbf{x})$ (mapping data to latent space). The loss function (negative log-likelihood) to minimize is:
\[
    \mathcal{L} = - \log p_{\mathbf{X}}(\mathbf{x}) = - \left( \log p_{\mathbf{Z}}(g^{-1}(\mathbf{x})) + \log \left| \det \frac{\partial g^{-1}(\mathbf{x})}{\partial \mathbf{x}} \right| \right)
\]
We update parameters to minimize $\mathcal{L}$ using gradient descent.

\textbf{2. Inference Algorithm (Sampling)} \\
To generate new data samples:
\begin{enumerate}
    \item Sample a latent vector from the prior distribution: $\mathbf{z} \sim p_{\mathbf{Z}}(z)$ (usually $\mathcal{N}(0, \mathbf{I})$).
    \item Transform it to data space using the forward network: $\mathbf{x} = g(\mathbf{z})$.
\end{enumerate}



\section*{Problem 5: Diffusion Models (20 points)}

% --- 题目 5(a) ---
\section*{Problem 5 (a): Forward Process}

The forward process (diffusion process) is a Markov chain that gradually adds Gaussian noise to the data according to a variance schedule $\beta_1, \dots, \beta_T$.

For a single step transition from $\mathbf{x}_{t-1}$ to $\mathbf{x}_t$:
\[
    q(\mathbf{x}_t | \mathbf{x}_{t-1}) = \mathcal{N}(\mathbf{x}_t; \sqrt{1 - \beta_t}\mathbf{x}_{t-1}, \beta_t \mathbf{I})
\]

The joint distribution for the entire forward process is:
\[
    q(\mathbf{x}_{1:T} | \mathbf{x}_0) = \prod_{t=1}^T q(\mathbf{x}_t | \mathbf{x}_{t-1})
\]

% --- 题目 5(b) ---
\section*{Problem 5 (b): Marginal Distribution Proof}

We define $\alpha_t = 1 - \beta_t$ and $\bar{\alpha}_t = \prod_{s=1}^t \alpha_s$.
Using the reparameterization trick, we can express $\mathbf{x}_t$ in terms of $\mathbf{x}_{t-1}$:
\[
    \mathbf{x}_t = \sqrt{\alpha_t}\mathbf{x}_{t-1} + \sqrt{1-\alpha_t}\epsilon_{t-1}, \quad \text{where } \epsilon_{t-1} \sim \mathcal{N}(0, \mathbf{I})
\]

We can expand this recursively. Substitute $\mathbf{x}_{t-1}$:
\[
\begin{aligned}
    \mathbf{x}_t &= \sqrt{\alpha_t}(\sqrt{\alpha_{t-1}}\mathbf{x}_{t-2} + \sqrt{1-\alpha_{t-1}}\epsilon_{t-2}) + \sqrt{1-\alpha_t}\epsilon_{t-1} \\
    &= \sqrt{\alpha_t \alpha_{t-1}}\mathbf{x}_{t-2} + \sqrt{\alpha_t(1-\alpha_{t-1})}\epsilon_{t-2} + \sqrt{1-\alpha_t}\epsilon_{t-1}
\end{aligned}
\]

Since the sum of two independent Gaussians $\mathcal{N}(0, \sigma_1^2) + \mathcal{N}(0, \sigma_2^2)$ is $\mathcal{N}(0, \sigma_1^2 + \sigma_2^2)$, the noise terms merge. By induction to $t=0$:
\[
    \mathbf{x}_t = \sqrt{\bar{\alpha}_t}\mathbf{x}_0 + \sqrt{1-\bar{\alpha}_t}\epsilon, \quad \epsilon \sim \mathcal{N}(0, \mathbf{I})
\]

Thus, the marginal distribution is:
\[
    q(\mathbf{x}_t | \mathbf{x}_0) = \mathcal{N}(\mathbf{x}_t; \sqrt{\bar{\alpha}_t}\mathbf{x}_0, (1 - \bar{\alpha}_t)\mathbf{I})
\]

% --- 题目 5(c) ---
\section*{Problem 5 (c): Posterior Distribution Proof}

We want to find $q(\mathbf{x}_{t-1} | \mathbf{x}_t, \mathbf{x}_0)$. Using Bayes' rule:
\[
    q(\mathbf{x}_{t-1} | \mathbf{x}_t, \mathbf{x}_0) = \frac{q(\mathbf{x}_t | \mathbf{x}_{t-1}) q(\mathbf{x}_{t-1} | \mathbf{x}_0)}{q(\mathbf{x}_t | \mathbf{x}_0)}
\]
Since all distributions are Gaussian, the posterior is also Gaussian. We focus on the exponent terms (completing the square for $\mathbf{x}_{t-1}$):

\[
\begin{aligned}
    & \log q(\mathbf{x}_{t-1} | \mathbf{x}_t, \mathbf{x}_0) \\
    &\propto -\frac{1}{2\beta_t} \|\mathbf{x}_t - \sqrt{\alpha_t}\mathbf{x}_{t-1}\|^2 - \frac{1}{2(1-\bar{\alpha}_{t-1})} \|\mathbf{x}_{t-1} - \sqrt{\bar{\alpha}_{t-1}}\mathbf{x}_0\|^2 + C
\end{aligned}
\]
By expanding the squares and matching coefficients with the standard Gaussian form $-\frac{1}{2\tilde{\beta}_t}\|\mathbf{x}_{t-1} - \tilde{\mu}_t\|^2$, we derive the mean $\tilde{\mu}_t$ and variance $\tilde{\beta}_t$:

\textbf{Variance $\tilde{\beta}_t$:}
The coefficient of $\mathbf{x}_{t-1}^2$ gives $\frac{1}{\tilde{\beta}_t} = \frac{1}{\beta_t} + \frac{1}{1-\bar{\alpha}_{t-1}}$, which simplifies to $\tilde{\beta}_t = \frac{1-\bar{\alpha}_{t-1}}{1-\bar{\alpha}_t} \beta_t$.

\textbf{Mean $\tilde{\mu}_t$:}
By matching the linear term of $\mathbf{x}_{t-1}$, we obtain:
\[
    \tilde{\mu}_t(\mathbf{x}_0, \mathbf{x}_t) = \frac{\sqrt{\alpha_t}(1-\bar{\alpha}_{t-1})}{1-\bar{\alpha}_t}\mathbf{x}_t + \frac{\sqrt{\bar{\alpha}_{t-1}}\beta_t}{1-\bar{\alpha}_t}\mathbf{x}_0
\]

% --- 题目 5(d) ---
\section*{Problem 5 (d): Training and Inference Algorithm}

\textbf{1. Training Algorithm (Learning to Denoise)}
We train a network $\epsilon_\theta$ to predict the noise added to the image.
\begin{itemize}
    \item Sample data $\mathbf{x}_0 \sim q(\mathbf{x}_0)$.
    \item Sample time step $t \sim \text{Uniform}(\{1, \dots, T\})$.
    \item Sample noise $\epsilon \sim \mathcal{N}(0, \mathbf{I})$.
    \item Compute noisy image: $\mathbf{x}_t = \sqrt{\bar{\alpha}_t}\mathbf{x}_0 + \sqrt{1-\bar{\alpha}_t}\epsilon$.
    \item \textbf{Minimize Loss (MSE):} $\| \epsilon - \epsilon_\theta(\mathbf{x}_t, t) \|^2$.
\end{itemize}

\textbf{2. Inference Algorithm (Sampling)}
We start from pure noise and iteratively remove it using the trained network.
\begin{itemize}
    \item Start with $\mathbf{x}_T \sim \mathcal{N}(0, \mathbf{I})$.
    \item For $t = T, \dots, 1$:
    \[
        \mathbf{z} \sim \mathcal{N}(0, \mathbf{I}) \text{ (if } t > 1 \text{ else } 0)
    \]
    \[
        \mathbf{x}_{t-1} = \frac{1}{\sqrt{\alpha_t}} \left( \mathbf{x}_t - \frac{1-\alpha_t}{\sqrt{1-\bar{\alpha}_t}} \epsilon_\theta(\mathbf{x}_t, t) \right) + \sigma_t \mathbf{z}
    \]
    \item Return $\mathbf{x}_0$.
\end{itemize}

\end{document}